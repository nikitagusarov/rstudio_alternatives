% Options for packages loaded elsewhere
\PassOptionsToPackage{unicode}{hyperref}
\PassOptionsToPackage{hyphens}{url}
%
\documentclass[
  11pt,
  ignorenonframetext,
  aspectratio=169]{beamer}
\usepackage{pgfpages}
\setbeamertemplate{caption}[numbered]
\setbeamertemplate{caption label separator}{: }
\setbeamercolor{caption name}{fg=normal text.fg}
\beamertemplatenavigationsymbolsempty
% Prevent slide breaks in the middle of a paragraph
\widowpenalties 1 10000
\raggedbottom
\setbeamertemplate{part page}{
  \centering
  \begin{beamercolorbox}[sep=16pt,center]{part title}
    \usebeamerfont{part title}\insertpart\par
  \end{beamercolorbox}
}
\setbeamertemplate{section page}{
  \centering
  \begin{beamercolorbox}[sep=12pt,center]{part title}
    \usebeamerfont{section title}\insertsection\par
  \end{beamercolorbox}
}
\setbeamertemplate{subsection page}{
  \centering
  \begin{beamercolorbox}[sep=8pt,center]{part title}
    \usebeamerfont{subsection title}\insertsubsection\par
  \end{beamercolorbox}
}
\AtBeginPart{
  \frame{\partpage}
}
\AtBeginSection{
  \ifbibliography
  \else
    \frame{\sectionpage}
  \fi
}
\AtBeginSubsection{
  \frame{\subsectionpage}
}
\usepackage{lmodern}
\usepackage{amssymb,amsmath}
\usepackage{ifxetex,ifluatex}
\ifnum 0\ifxetex 1\fi\ifluatex 1\fi=0 % if pdftex
  \usepackage[T1]{fontenc}
  \usepackage[utf8]{inputenc}
  \usepackage{textcomp} % provide euro and other symbols
\else % if luatex or xetex
  \usepackage{unicode-math}
  \defaultfontfeatures{Scale=MatchLowercase}
  \defaultfontfeatures[\rmfamily]{Ligatures=TeX,Scale=1}
\fi
\usetheme[]{Montpellier}
\usecolortheme{beaver}
% Use upquote if available, for straight quotes in verbatim environments
\IfFileExists{upquote.sty}{\usepackage{upquote}}{}
\IfFileExists{microtype.sty}{% use microtype if available
  \usepackage[]{microtype}
  \UseMicrotypeSet[protrusion]{basicmath} % disable protrusion for tt fonts
}{}
\makeatletter
\@ifundefined{KOMAClassName}{% if non-KOMA class
  \IfFileExists{parskip.sty}{%
    \usepackage{parskip}
  }{% else
    \setlength{\parindent}{0pt}
    \setlength{\parskip}{6pt plus 2pt minus 1pt}}
}{% if KOMA class
  \KOMAoptions{parskip=half}}
\makeatother
\usepackage{xcolor}
\IfFileExists{xurl.sty}{\usepackage{xurl}}{} % add URL line breaks if available
\IfFileExists{bookmark.sty}{\usepackage{bookmark}}{\usepackage{hyperref}}
\hypersetup{
  pdftitle={Alternatives to RStudio},
  hidelinks,
  pdfcreator={LaTeX via pandoc}}
\urlstyle{same} % disable monospaced font for URLs
\newif\ifbibliography
\setlength{\emergencystretch}{3em} % prevent overfull lines
\providecommand{\tightlist}{%
  \setlength{\itemsep}{0pt}\setlength{\parskip}{0pt}}
\setcounter{secnumdepth}{-\maxdimen} % remove section numbering
% Colors
\usepackage{colortbl}

% Change section names style
\titlegraphic{\flushright\includegraphics[height = 8mm]{../images/gael_logo_3.jpg}
  \includegraphics[height = 8mm]{../images/gscop_logo.jpg}
  \includegraphics[height = 8mm]{../images/inp_logo.png} 
}

% Change argmax / argmin looks
\usepackage{amsmath}
\DeclareMathOperator*{\argmax}{arg\,max}
\DeclareMathOperator*{\argmin}{arg\,min}

% Background images 
% \usepackage[pages = some, scale = 1, angle = 0, opacity = 1]{background}

% Floats
\usepackage{placeins}

% Multiple rows
\usepackage{multirow}

% Multiple columns of text
\usepackage{multicol}

% Combining cells
\usepackage{makecell}

% Include schemas and tikz graphics
\usepackage{tikz}
\usetikzlibrary{er,positioning}

% Margins
\newcommand{\fullframegraphic}[1]{
  \includegraphics[height=\paperheight,width=\paperwidth,keepaspectratio]{#1}
}
\makeatletter
\newlength\beamerleftmargin
\setlength\beamerleftmargin{\Gm@lmargin}
\makeatother

% Add footline
\makeatletter
\setbeamertemplate{footline}{%
  \leavevmode%
  \hbox{%
  \begin{beamercolorbox}[wd=.333333\paperwidth,ht=2.25ex,dp=1ex,center]{author in head/foot}%
    \usebeamerfont{author in head/foot}{Nikita Gusarov}
  \end{beamercolorbox}%
  \begin{beamercolorbox}[wd=.333333\paperwidth,ht=2.25ex,dp=1ex,center]{title in head/foot}%
    \usebeamerfont{institute in head/foot}\insertshortdate
  \end{beamercolorbox}%
  \begin{beamercolorbox}[wd=.333333\paperwidth,ht=2.25ex,dp=1ex,right]{date in head/foot}%
    \usebeamerfont{date in head/foot}\insertshortinstitute{}\hspace*{1em}
   %\insertframenumber{} / \inserttotalframenumber\hspace*{2ex} % old version
    \insertframenumber{} \hspace*{2ex} % new version without total frames
  \end{beamercolorbox}}%
  \vskip0pt%
}
\makeatother

\title{Alternatives to RStudio}
\subtitle{Workflow organisation with R}
\author{Nikita Gusarov\\
\scriptsize GAEL (UGA) - G-SCOP (Grenoble INP)}
\date{17/12/2020}

\begin{document}
\frame{\titlepage}

\hypertarget{introduction}{%
\section{Introduction}\label{introduction}}

\begin{frame}{Workflow}
\protect\hypertarget{workflow}{}
\begin{center}\includegraphics{presentation_files/figure-beamer/unnamed-chunk-2-1} \end{center}
\end{frame}

\begin{frame}{Workflow types}
\protect\hypertarget{workflow-types}{}
\begin{itemize}
\tightlist
\item
  Text editor
\item
  Notebook
\item
  Integrated Development Environment (IDE)
\end{itemize}
\end{frame}

\begin{frame}{}
\protect\hypertarget{section}{}
\vspace*{-10.6mm}\hspace*{-\beamerleftmargin}\hspace*{-1.5mm}
\fullframegraphic{"../captures/Capture d’écran (0)"}
\end{frame}

\begin{frame}{}
\protect\hypertarget{section-1}{}
\vspace*{-10.6mm}\hspace*{-\beamerleftmargin}\hspace*{-1.5mm}
\fullframegraphic{"../captures/Capture d’écran (1)"}
\end{frame}

\begin{frame}{}
\protect\hypertarget{section-2}{}
\vspace*{-10.6mm}\hspace*{-\beamerleftmargin}\hspace*{-1.5mm}
\fullframegraphic{"../captures/Capture d’écran (3)"}
\end{frame}

\hypertarget{what-is-rstudio}{%
\section{What is RStudio ?}\label{what-is-rstudio}}

\begin{frame}{}
\protect\hypertarget{section-3}{}
\vspace*{-10.6mm}\hspace*{-\beamerleftmargin}\hspace*{-1.5mm}
\fullframegraphic{"../captures/Capture d’écran (2)"}
\end{frame}

\begin{frame}{``An IDE that was built for R''}
\protect\hypertarget{an-ide-that-was-built-for-r}{}
\begin{itemize}
\tightlist
\item
  Syntax highlighting, code completion, and smart indentation
\item
  Execute R code directly from the source editor
\item
  Quickly jump to function definitions
\end{itemize}
\end{frame}

\begin{frame}{``Helps bring your workflow together''}
\protect\hypertarget{helps-bring-your-workflow-together}{}
\begin{itemize}
\tightlist
\item
  Integrated R help and documentation
\item
  Easily manage multiple working directories using projects
\item
  Workspace browser and data viewer
\end{itemize}
\end{frame}

\begin{frame}{``Powerful authoring and debugging''}
\protect\hypertarget{powerful-authoring-and-debugging}{}
\begin{itemize}
\tightlist
\item
  Interactive debugger to diagnose and fix errors quickly
\item
  Extensive package development tools
\item
  Authoring with Sweave and R Markdown
\end{itemize}
\end{frame}

\begin{frame}{And more \ldots{}}
\protect\hypertarget{and-more}{}
\begin{itemize}
\tightlist
\item
  Runs on most desktops or on a server and accessed over the web
\item
  Integrates the tools you use with R into a single environment
\item
  Includes powerful coding tools designed to enhance your productivity
\item
  Enables rapid navigation to files and functions
\end{itemize}
\end{frame}

\begin{frame}{And more \ldots{}}
\protect\hypertarget{and-more-1}{}
\begin{itemize}
\tightlist
\item
  Makes it easy to start new or find existing projects
\item
  Has integrated support for Git and Subversion
\item
  Supports authoring HTML, PDF, Word Documents, and slide shows
\item
  Supports interactive graphics with Shiny and ggvis
\end{itemize}
\end{frame}

\begin{frame}{}
\protect\hypertarget{section-4}{}
\vspace*{-10.6mm}\hspace*{-\beamerleftmargin}\hspace*{-1.5mm}
\fullframegraphic{"../captures/Capture d’écran (3)"}
\end{frame}

\hypertarget{what-rstudio-is-not}{%
\section{What RStudio is not \ldots{}}\label{what-rstudio-is-not}}

\begin{frame}{Not a multi-language DE}
\protect\hypertarget{not-a-multi-language-de}{}
\vspace*{-10.6mm}\hspace*{-\beamerleftmargin}\hspace*{-1.5mm}
\fullframegraphic{"../captures/Capture d’écran (12)"}
\end{frame}

\hypertarget{what-about-alternatives}{%
\section{What about alternatives ?}\label{what-about-alternatives}}

\hypertarget{jupyter}{%
\section{Jupyter}\label{jupyter}}

\begin{frame}{}
\protect\hypertarget{section-5}{}
\vspace*{-10.6mm}\hspace*{-\beamerleftmargin}\hspace*{-1.5mm}
\fullframegraphic{"../captures/Capture d’écran (15)"}
\end{frame}

\begin{frame}{Key advertising factors}
\protect\hypertarget{key-advertising-factors}{}
\begin{itemize}
\tightlist
\item
  A web application
\end{itemize}

\begin{itemize}
\tightlist
\item
  Notebook documents
\end{itemize}
\end{frame}

\begin{frame}{Features}
\protect\hypertarget{features}{}
\begin{itemize}
\tightlist
\item
  In-browser editing for code
\item
  Automatic syntax highlighting, indentation, and tab completion
\item
  The ability to execute code from the browser
\item
  Results of computations is attached to the code which generated them
\end{itemize}
\end{frame}

\begin{frame}{Features}
\protect\hypertarget{features-1}{}
\begin{itemize}
\tightlist
\item
  Displaying the result of computation using rich media representations
  (HTML, LaTeX, PNG, SVG, etc)
\item
  In-browser editing for rich text using the Markdown markup language
\item
  Commentary for the code is not limited to plain text
\item
  Include mathematical notation within markdown cells using LaTeX, and
  rendered natively by MathJax.
\end{itemize}
\end{frame}

\begin{frame}{}
\protect\hypertarget{section-6}{}
\vspace*{-10.6mm}\hspace*{-\beamerleftmargin}\hspace*{-1.5mm}
\fullframegraphic{"../captures/Capture d’écran (1)"}
\end{frame}

\begin{frame}{Drawbacks}
\protect\hypertarget{drawbacks}{}
\end{frame}

\begin{frame}{Version control}
\protect\hypertarget{version-control}{}
The .ipynb Jupyter Notebook files are blobs of JSON that also store cell
output as well as metadata.
\end{frame}

\begin{frame}{}
\protect\hypertarget{section-7}{}
\vspace*{-10.6mm}\hspace*{-\beamerleftmargin}\hspace*{-1.5mm}
\fullframegraphic{"../captures/Capture d’écran (14)"}
\end{frame}

\begin{frame}{Inline Code Rendering}
\protect\hypertarget{inline-code-rendering}{}
In Jupyter Notebooks, it is impossible to use the inline expressions
without additional markdown modules.
\end{frame}

\hypertarget{vs-code}{%
\section{VS Code}\label{vs-code}}

\begin{frame}{}
\protect\hypertarget{section-8}{}
\vspace*{-10.6mm}\hspace*{-\beamerleftmargin}\hspace*{-1.5mm}
\fullframegraphic{"../captures/Capture d’écran (4)"}
\end{frame}

\begin{frame}{VS Code selling features}
\protect\hypertarget{vs-code-selling-features}{}
\begin{itemize}
\tightlist
\item
  Simplicity of a source code editor
\item
  Powerful developer tooling (IntelliSense, code completion and
  debugging)
\end{itemize}
\end{frame}

\begin{frame}{Available for macOS, Linux, and Windows}
\protect\hypertarget{available-for-macos-linux-and-windows}{}
Visual Studio Code supports macOS, Linux, and Windows - so you can hit
the ground running, no matter the platform.
\end{frame}

\begin{frame}{``Edit, build, and debug with ease''}
\protect\hypertarget{edit-build-and-debug-with-ease}{}
\begin{itemize}
\tightlist
\item
  Lightning fast source code editor
\item
  Support for hundreds of languages
\item
  Intuitive keyboard shortcuts, easy customization and
  community-contributed mappings
\item
  Interactive debugger
\item
  Build and scripting tools to perform common tasks
\item
  Support for Git so you can work with source control without leaving
  the editor including viewing pending changes diffs
\end{itemize}
\end{frame}

\begin{frame}{``Make it your own''}
\protect\hypertarget{make-it-your-own}{}
\begin{itemize}
\tightlist
\item
  Customization through extensions
\item
  Open-source project
\end{itemize}
\end{frame}

\begin{frame}{}
\protect\hypertarget{section-9}{}
\vspace*{-10.6mm}\hspace*{-\beamerleftmargin}\hspace*{-1.5mm}
\fullframegraphic{"../captures/Capture d’écran (5)"}
\end{frame}

\begin{frame}{}
\protect\hypertarget{section-10}{}
\vspace*{-10.6mm}\hspace*{-\beamerleftmargin}\hspace*{-1.5mm}
\fullframegraphic{"../captures/Capture d’écran (6)"}
\end{frame}

\begin{frame}{}
\protect\hypertarget{section-11}{}
\vspace*{-10.6mm}\hspace*{-\beamerleftmargin}\hspace*{-1.5mm}
\fullframegraphic{"../captures/Capture d’écran (8)"}
\end{frame}

\begin{frame}{}
\protect\hypertarget{section-12}{}
\vspace*{-10.6mm}\hspace*{-\beamerleftmargin}\hspace*{-1.5mm}
\fullframegraphic{"../captures/Capture d’écran (13)"}
\end{frame}

\begin{frame}{References}
\protect\hypertarget{references}{}
\small

\begin{itemize}
\tightlist
\item
  \href{https://insights.stackoverflow.com/survey/2019\#technology-_-most-popular-development-environments}{StackOverflow
  survey}
\item
  \href{https://code.visualstudio.com/docs/editor/whyvscode}{Visual
  Studio Code}
\item
  \href{https://rstudio.com/products/rstudio/features/}{RStudio and
  features}
\item
  \href{https://forums.fast.ai/t/text-editor-vs-jupyter-notebook-vs-ides/22069}{Discussion
  of editors}
\item
  \href{https://stackshare.io/stackups/jupyter-vs-rstudio}{Jupyter
  against RStudio}
\item
  \href{https://stackshare.io/stackups/rstudio-vs-visual-studio-code}{VS
  Code against RStudio}
\item
  \href{https://renkun.me/2019/12/11/writing-r-in-vscode-a-fresh-start/}{Starting
  with R in VS Code}
\item
  \href{https://www.makeuseof.com/tag/visual-studio-code-vs-atom/}{VS
  Code and Atom}
\end{itemize}
\end{frame}

\end{document}
